\documentclass[10pt,a4paper,sans]{moderncv}

% moderncv themes
\moderncvstyle{classic}
\moderncvcolor{blue}

% character encoding
\usepackage[utf8]{inputenc}

% adjust the page margins
\usepackage[scale=0.80]{geometry}
\setlength{\hintscolumnwidth}{3.1cm} % the width of the column with the dates

% personal data
\name{Benjamin}{Collet}
\title{Ingénieur en recherche opérationnelle}
\address{46 Rue du Vieux Pont}{01550 Pougny, France}{}
\phone[mobile]{+33~6~95~36~00~58}
\email{benjamin.collet@protonmail.ch}
\social[linkedin][www.linkedin.com/in/benjamin-collet]{linkedin.com/in/benjamin-collet}
\social[github][github.com/TheMrBen]{github.com/TheMrBen}
\photo[64pt][0.4pt]{portrait}

%----------------------------------------------------------------------------------
%            content
%----------------------------------------------------------------------------------
\begin{document}
\makecvtitle

\section{Stages}
\cventry{Février -- Juillet 2019}{LocalSolver}{Stage de fin d'études}{}{}{}
    \cvlistitem{Calcul de bornes inférieures pour des tournées de véhicules (génération de colonnes)}
    \cvlistitem{Réécriture du système de licences pour un modèle client--serveur}
    \vspace{7pt}
\cventry{Février -- Juin 2018}{Verimag (laboratoire de recherche)}{Travail Encadré de Recherche}{}{}{}
    \cvlistitem{Modélisation d'une topologie de réseau avec l'assistant de preuve Coq}
    \vspace{7pt}
\cventry{Avril -- Juin 2016}{LISTIC (laboratoire de recherche)}{Stage de DUT}{}{}{}
    \cvlistitem{Programmation d’un parseur entre deux langages mathématique}
    \cvlistitem{Création d’une interface graphique complète et ergonomique}

\section{Formation}
\cventry{2017 -- 2019}{Master}{Université Grenoble Alpes}{}{}{Master of Science in Informatics at Grenoble (MoSIG)}
    \cvlistitem{Spécialité <<~Recherche Opérationnelle, Combinatoire et Optimisations~>>}
    \vspace{7pt}
\cventry{2016 -- 2017}{Licence Erasmus}{Oulu University of Applied Sciences}{Finlande}{}{Information and Communications Technology}
    \cvlistitem{Études à l'étranger~: anglais, \textbf{adaptation} à la culture, \textbf{autonomie}}
    \cvlistitem{Projet d'entreprises, \textbf{travail en équipe}}
    \vspace{7pt}
\cventry{2014 -- 2016}{Diplôme Universitaire de Technologie}{IUT d'Annecy}{}{}{Informatique}
    \cvlistitem{Conception de logiciels et de bases de données}
    \cvlistitem{Introduction à l'économie, la gestion et la communication}

\section{Expériences professionnelles}
\cventry{Été 2013 à 2015}{Entretien extérieur}{Jardicrêt, puis SFMCP (centrale hydroélectrique)}{France, puis Suisse}{}{}
    \cvlistitem{\textbf{Travail d’équipe}, besoin d’\textbf{efficacité} car petite entreprise}
    \cvlistitem{\textbf{Respect des règles de sécurité} car utilisation de matériels dangereux}

\section{Langues}
\cvitemwithcomment{Anglais}{Courant (C1)}{études à l'étranger (Finlande)}

\section{Centres d'intérêt}
\cvitem{Association étudiante}{Impliqué dans une association qui accueille les étudiants internationaux. J'ai amélioré mes compétences en \textbf{communication interpersonnelle} et \textbf{improvisation}.}
    \vspace{7pt}
\cvitem{Sport}{Ceinture noire de Kung Fu, 8 ans de pratique et un voyage en Chine. Cela m’a apporté de la \textbf{discipline}, de la \textbf{rigueur} et de la \textbf{persévérance}.}

\end{document}
